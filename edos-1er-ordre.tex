% Handouts
%
% Tema 3. Sistemes Lineals. Mètodes Iteratius
%

\documentclass[slidetop,compress,10pt]{beamer}

\mode<presentation>{\usetheme{SimplePlus}}
\setbeamertemplate{theorems}[numbered]
\setbeamertemplate{caption}[numbered]
\setbeamertemplate{itemize item}[triangle]
\setbeamertemplate{itemize subitem}[circle]
\setbeamertemplate{frametitle continuation}{%
\usebeamerfont{frametitle}
(\insertcontinuationcountroman)
}

%\apptocmd{\frame}{}{\justifying}{} % Allow optional arguments after frame.

\usepackage{mathptmx}
\usepackage[scaled=0.9]{helvet}
\usepackage{courier}

\usepackage{ragged2e}
\usepackage{amsmath,amssymb}
\usepackage{amsfonts}
\usepackage{mathtools}
\usepackage{wrapfig}
\usepackage{cancel}
\usepackage{fontawesome}
\usepackage{caption}
\usepackage{color}

\usepackage[utf8]{inputenc}
\usepackage[catalan]{babel}
\selectlanguage{catalan}
\uselanguage{Catalan}
\languagepath{Catalan}

\usepackage{enumitem}

\usepackage{ifpdf}
\usepackage{numprint}
\usepackage{oldstyle}
\graphicspath{ {_figures/} }
\DeclareGraphicsExtensions{.png}
\usepackage{subfigure}

\usepackage{yhmath}

\usepackage[round]{natbib}

\hypersetup{
  pdfauthor   = {J.R. Pacha},
  pdftitle    = {Calcul2}
  pdfsubject  = {EDOs},
  pdfpagemode = {FullScreen},
  pdfstartview = {},
  colorlinks  = {true},
%  bookmarks   = {true},
%  pagebackref = {true},
  bookmarksnumbered = {true},
  hyperindex  = {true}
}
% \pdfadjustspacing=1

\usepackage{url}
%\setcounter{section}{3}

% THEOREMS -------------------------------------------------------
\newtheorem{thm}{Teorema}%[section]
\newtheorem{cor}[thm]{Corollary}
\newtheorem{lem}[thm]{Lemma}
\theoremstyle{definition}
\newtheorem{prop}{Proposici{\'{o}}}%[thm]{Proposició}
\newtheorem{defn}{Definici{\'{o}}}%[thm]{Definici\'{o}}
\theoremstyle{remark}
%\newtheorem{rem}[thm]{Remarca}
\newtheorem{rem}{Remarca}
%\newtheorem{ex}[thm]{Exemple}
\newtheorem{ex}{Exemple}
%\numberwithin{equation}{section}
\newtheorem{exc}{Exercici}
\newtheorem{demo}{Demostraci{\'{o}}}
\newtheorem*{demo*}{Demostraci{\'{o}}}

% Algorismes -----------------------------------------------------
\usepackage[noline,plain]{algorithm2e}

\setlength{\AlCapSkip}{0.7em}

\usepackage{hyperref}

% MATH -----------------------------------------------------------
\newcommand{\norm}[1]{\left\Vert#1\right\Vert}
\newcommand{\abs}[1]{\left\vert#1\right\vert}
\newcommand{\set}[1]{\left\{#1\right\}}
\newcommand{\bs}[1]{\ensuremath{\boldsymbol{#1}}}
\newcommand{\C}{\ensuremath{\mathbb{C}}}
\newcommand{\R}{\ensuremath{\mathbb{R}}}
\newcommand{\N}{\ensuremath{\mathbb{N}}}
\newcommand{\M}{\ensuremath{\mathbb{M}}}
\newcommand{\Z}{\ensuremath{\mathbb{Z}}}
\newcommand{\K}{\ensuremath{\mathbb{K}}}
\newcommand{\eps}{\ensuremath{\epsilon}}
\newcommand{\veps}{\ensuremath{\varepsilon}}
\newcommand{\To}{\longrightarrow}
\newcommand{\BX}{\mathbf{B}(X)}
\newcommand{\A}{\mathcal{A}}
\newcommand{\Or}{\mathcal{O}}
\newcommand{\Id}[1]{Id_{#1}}
\newcommand{\argcosh}{\ensuremath{\mathrm{argcosh}}\,}
\newcommand{\argsinh}{\ensuremath{\mathrm{argsinh}}\,}
\newcommand{\argtanh}{\ensuremath{\mathrm{argtanh}}\,}
\newcommand{\argcoth}{\ensuremath{\mathrm{argcoth}}\,}
\newcommand{\rme}{\mathrm{e}}
\newcommand{\rmi}{\mathrm{i}}
\newcommand{\rmd}{\mathrm{d}}
\newcommand{\re}{\mathrm{Re\:}}
\newcommand{\im}{\mathrm{Im\:}}
\newcommand{\wt}[1]{\ensuremath{\widetilde{#1}}}
\DeclareMathOperator{\Hessian}{Hess}
\DeclareMathOperator*{\maxc}{\text{\rm m\`{a}x}}
\DeclareMathOperator*{\limc}{\text{\rm l\'{\i}m}}
\DeclareMathOperator*{\minc}{\text{\rm m\'{\i}n}}
\newcommand{\interior}[1]{\ensuremath{\mathring{#1}}}
\DeclareMathOperator{\ext}{Ext}
\DeclareMathOperator{\fr}{Fr}
\DeclareMathOperator{\Dom}{\mathcal{D}}
\newcommand{\dom}[1]{\Dom\left(#1\right)}
\DeclareMathOperator{\Rang}{\mathcal{R}}
\newcommand{\rang}[1]{\Rang\left(#1\right)}
\newcommand{\Mat}[2]{\ensuremath M_{#1}(#2)}
\newcommand{\bracket}[2]{\ensuremath\left\langle #1, #2\right\rangle}
\newcommand{\verteq}{\rotatebox{90}{$\scriptstyle \,=$}}
\newcommand{\eqover}[1]{\substack{\verteq \\ \scriptstyle #1}}
\newcommand{\equnder}[1]{\substack{\scriptstyle #1 \\ \verteq}}

% ----------------------------------------------------------------
\newcommand{\gl}{\guillemotleft}
\newcommand{\gr}{\guillemotright}
\newcommand{\sst}{\ensuremath{\scriptstyle}}
\newcommand{\npos}[2]{\ensuremath{\oldstyle{\numprint[#1]{#2}}}}
\renewcommand{\thefootnote}{\fnsymbol{footnote}} %{$\scriptstyle (*)$}

% Counters -------------------------------------------------------
\newcounter{saveenum}
\beamersetuncovermixins{\opaqueness<1>{25}}{\opaqueness<2->{15}}
\setbeamertemplate{caption}[numbered]
\setbeamertemplate{enumerate item}{\arabic{enumi})}%
\setbeamertemplate{enumerate label}{\roman{enumi})}
\setbeamerfont{caption}{size=\scriptsize}

\newdimen\oldparindent
\oldparindent=\parindent  %Keep the original parindent

\AtBeginSection[]{
  \begin{frame}
    \vfill
    \centering
    \begin{beamercolorbox}[sep=8pt, center, shadow=true,
      rounded=true]{title}
      \usebeamerfont{title}\insertsectionhead\par%
    \end{beamercolorbox}
    \vfill
  \end{frame}
}

%\title[]%
%{{\Large Càlcul II (240022)} \\
%\large Tema 4. Equacions Diferencials de $1^{\text{er}}$ ordre}
%
%\date{\today}

\title{Càlcul II (240022)\\
Tema 4. Equacions Diferencials Ordinàries (EDOs) de primer ordre}
\author{}
\institute{}
%\insertshortdate

%%----------------------------------------------------------------------------
\begin{document}
%\frame{\titlepage}
%\frame{\tableofcontents}

\begin{frame}
  \maketitle
\end{frame}

%%----------------------------------------------------------------------------
\section{EDOs Lineals}
\begin{frame}[t,allowframebreaks]%{EDOs de 1er Ordre}%
\justifying
%\citep[In][chap.~11]{Reddy2006}
\begin{defn}\justifying
Siguin $I\subseteq\R$ un interval de $\R$ i $a, b: I\longrightarrow\R$
funcions contínues definides en $I$ interval de $\R$. L'equació
\begin{equation}
  y' + a(x) y = b(x)
  \label{eq:edo-lineal-1er-ordre}
\end{equation}
és una equació diferencial ordinària (EDO) de $1^{\text{er}}$ ordre on $a$
és el coeficient i $b$ el terme independent.
\end{defn}

\begin{defn}\justifying
  Direm que una funció $\phi: I\longrightarrow\R$, $\phi\in
  \mathcal{C}^{1}(I)$, és una solució (o una solució \emph{particular}) de
  l'EDO~\eqref{eq:edo-lineal-1er-ordre}, si
  \begin{displaymath}
    \phi'(x) + a(x)\phi(x) = b(x)
  \end{displaymath}
  per a cada $x\in I$.
\end{defn}

\begin{example}\label{ex:edo-lineal-1er-ordre}\justifying
  La funció $y(x) = x^{2} / 3$ és una solució particular, definida a
  l'interval $I = (0, +\infty)$, de l'EDO lineal de
  $1^{\text{er}}$ ordre,
  \begin{equation}
    y' + y / x = x.
    \label{eq:edo-lineal-1er-ordre-1}
  \end{equation}
\end{example}
\end{frame}

\begin{frame}[t]\justifying
  \begin{defn}\justifying
    Direm que una família de funcions uniparamètrica
    \begin{displaymath}
      \mathcal{F} = \left\{
        y(\cdot, C): I\longrightarrow\R, C\in\mathcal{A}\subseteq\R
      \right\},
    \end{displaymath}
    $Y(\cdot, C)\in \mathcal{C}^{1}(I)$ per a cada $C\in\mathcal{A}$,
    és la \textbf{solució general} de l'EDO~\eqref{eq:edo-lineal-1er-ordre}
    si, per a tota solució $\phi_{\ast}$ de~\eqref{eq:edo-lineal-1er-ordre},
      existeix $C_{\ast}\in\mathcal{A}$ tal que $\phi_{\ast} = Y(\cdot,
      C_{\ast})$.
  \end{defn}

  \begin{example}\justifying
    La família de funcions
    \begin{displaymath}
      \arraycolsep = 1.1pt
      \begin{array}{rcl}
        Y(\cdot,C) : I = (0,+\infty) & \longrightarrow & \R\\
        x\in I & \longrightarrow & y(x,C) = \dfrac{C}{x} + \dfrac{x^{2}}{3},
      \end{array}
    \end{displaymath}
    $C\in\R$, és la solució general de
    l'EDO~\eqref{eq:edo-lineal-1er-ordre-1}.
  \end{example}
  \begin{rem}
    Notem que la solució de l'EDO~\eqref{eq:edo-lineal-1er-ordre-1} donada a
    l'exemple~\ref{ex:edo-lineal-1er-ordre} correspon al valor $C = 0$ del
    paràmetre.
  \end{rem}
\end{frame}

\begin{frame}[t]
  \begin{prop}[Existència i unicitat de solució]\justifying
Donat el problema de valors inicials (PVI o de Cauchy) definit per l'EDO
lineal de $1^{\text{er}}$ ordre~\eqref{eq:edo-lineal-1er-ordre} i la
condició inicial $y(x_{0}) = y_{0}$, i.e.,
 \begin{equation}
   \left.
 \begin{aligned}
   y' + a(x) y &= b(x)\\
   y(x_{0}) &= y_{0}
 \end{aligned}
   \right\}\label{eq:pvi-lineal}
\end{equation}
amb $x_{0}\in\mathring{\!\! I}$, $y_{0}\in\R$ fixats, existeix una
\textbf{única} funció $\varphi: I\longrightarrow\R$ tal que:
\begin{enumerate}[label = (\roman*), ref = (\roman*)]
  \item $\varphi\in C^{1}(I)$,
  \item $\varphi'(x) + a(x)\varphi(x) = b(x)$ per a tot $x\in I$,
  \item $\varphi(x_{0}) = y_{0}$.
\end{enumerate}
Direm que $\varphi(x)$ és la solució del PVI~\eqref{eq:pvi-lineal}.
\end{prop}

\begin{defn}\justifying
  L'EDO lineal de $1^{\text{er}}$ ordre,
  \begin{equation}
    y' + a(x) y = 0,
    \label{eq:edo-lineal-homogenia-associada}
  \end{equation}
  s'anomena \emph{EDO linal homogènia associada a l'EDO lineal de
  $1^{\text{er}}$ ordre}~\eqref{eq:edo-lineal-1er-ordre}.
\end{defn}
\end{frame}

%%----------------------------------------------------------------------------
\begin{frame}[t]\justifying
\begin{rem}
  Notem que l'EDO homogènia~\eqref{eq:edo-lineal-homogenia-associada}
  s'obté com un cas particular de l'EDO lineal de primer
  ordre~\eqref{eq:edo-lineal-1er-ordre} agafant $b\equiv 0$.
\end{rem}

\begin{prop}
  La solució general de l'EDO lineal homogènia de $1^{\text{er}}$
  ordre~\eqref{eq:edo-lineal-homogenia-associada} ve donada per la família
  de funcions
  \begin{equation}
    y_{h}(x,C) = C\mathrm{e}^{A(x)},
    \label{eq:sol-edo-homogenia}
  \end{equation}
  $C\in\R$, on
  \begin{displaymath}
    \arraycolsep = 1.1pt
    \begin{array}{rrcl}
      & A: I & \longrightarrow & \R\\
        & x & \longrightarrow & A(x) =\displaystyle -\int a(x)\mathrm{d}x
    \end{array}
  \end{displaymath}
  és una primitiva \emph{qualsevol} de la funció $-a(x)$. Per tant
  \begin{displaymath}
    A'(x) = \frac{\mathrm{d}}{\mathrm{d} x}\left(-\int a(x)\mathrm{d}
      x\right) = -a(x)
  \end{displaymath}
  per a cada $x\in I$. Sovint~\eqref{eq:sol-edo-homogenia} s'escriu com
  \begin{equation}
    y_{h}(x, C) = C\exp\left(-\displaystyle\int a(x)\mathrm{d} x\right).
    \label{eq:sol-edo-homogenia-int}
  \end{equation}
\end{prop}

\end{frame}

%%----------------------------------------------------------------------------
\begin{frame}\justifying
  % Put here the content of the slide
  \begin{prop}\justifying
     La solució general de l'EDO lineal de $1^{er}$
     ordre~\eqref{eq:edo-lineal-1er-ordre} ve donada per la suma
     \begin{equation}
       y(x,C) = y_{h}(x,C) + y_{p} (x),\quad C\in\R,
       \label{eq:sol-edo-lineal-1er-ordre}
     \end{equation}
     de
     \begin{enumerate}[label = (\alph*), ref = (\alph*)]
       \item  $y_{h}(x,C)$, la solució
         general, \eqref{eq:sol-edo-homogenia}
         ó~\eqref{eq:sol-edo-homogenia-int}, de l'EDO lineal
         homogènia~\eqref{eq:edo-lineal-homogenia-associada}, associada
         a l'EDO lineal~\eqref{eq:edo-lineal-1er-ordre}, i
       \item $y_{p}(x)$, \emph{una} solució particular qualsevol de
         l'EDO~\eqref{eq:edo-lineal-1er-ordre}
     \end{enumerate}
  \end{prop}

  \begin{rem}\justifying
    De fet, 
    \begin{enumerate}[label=(\roman*), ref=(\roman*)]
      \item Fins ara només tenim fórmules explícites,
        \eqref{eq:sol-edo-homogenia} ó~\eqref{eq:sol-edo-homogenia-int} que
        ens donen la solució general, $y_{h}(x, C)$, de l'EDO lineal
        homogènia~\eqref{eq:edo-lineal-homogenia-associada} associada l'EDO
        lineal~\eqref{eq:edo-lineal-1er-ordre}.
      \item Per trobar una solució particular, $y_{p}(x)$, de
        l'EDO~\eqref{eq:edo-lineal-1er-ordre} i completar així la
        suma~\eqref{eq:sol-edo-lineal-1er-ordre} es pot fer servir el
        \emph{mètode de variació dels paràmetres} (o~\emph{mètode de
          variació de les constants}).
    \end{enumerate}
  \end{rem}
\end{frame}


%%----------------------------------------------------------------------------
\begin{frame}[t]{}\justifying
  % Put here the content of the  slide
  \begin{block}{Mètode de  variació dels paràmetres (Lagrange)}\justifying
    (O mètode de variació de les constants). Ens dóna un ``procediment'' per
    construir solucions particulars de l'EDO lineal de $1^{\text{er}}$
    ordre~\eqref{eq:edo-lineal-1er-ordre}, el qual consisteix en buscar
    solucions de la forma,
    \begin{equation}
      y_{p}(x) = h(x)\mathrm{e}^{A(x)},
      \label{eq:yp-vc}
    \end{equation}
    on $h(x)$ és una funció que fixarem substituint~\eqref{eq:yp-vc}
    a~\eqref{eq:edo-lineal-1er-ordre}:
    \begin{multline*}
      y'_{p}(x) + a(x) y_{p}(x) = h'(x)\mathrm{e}^{A(x)} + 
      A'(x)h(x)\mathrm{e}^{A(x)} + a(x) h(x)\mathrm{e}^{A(x)}\\
      = h'(x)\mathrm{e}^{h(x)} - \cancel{a(x) h(x)\mathrm{e}^{A(x)}}
      + \cancel{a(x) h(x)\mathrm{e}^{A(x)}} = h'(x)\mathrm{e}^{A(x)} =
      b(x)\\
      \Longleftrightarrow h'(x) = b(x)\mathrm{e}^{-A(x)}
      \Longleftrightarrow h(x) = \int  b(x)\mathrm{e}^{-A(x)}\mathrm{d} x;
    \end{multline*}
    i.e.~\eqref{eq:yp-vc} és solució de~\eqref{eq:edo-lineal-1er-ordre}
    sii $h(x)$ és una primitiva de la funció
    $b(x)\mathrm{e}^{A(x)}$. Aleshores,
    \begin{displaymath}
      y_{p}(x) = \mathrm{e}^{A(x)} \int  b(x)\mathrm{e}^{-A(x)}\mathrm{d} x
    \end{displaymath}
    i, aplicant~\eqref{eq:sol-edo-lineal-1er-ordre} 
    %s'arriba a la fórmula explícita següent per a la solució general
    %de~\eqref{eq:edo-lineal-1er-ordre}:
    \begin{displaymath}
      y(x,C) = y_{h}(x, C) + y_{p}(x) = C\mathrm{e}^{A(x)} +
      \mathrm{e}^{A(x)} \int  b(x)\mathrm{e}^{-A(x)}\mathrm{d} x,\quad
      C\in\R.
    \end{displaymath} 
  \end{block}

\end{frame}

%%----------------------------------------------------------------------------
\section{Exemple (problema 2 de la col·lecció)}

\begin{frame}[t]\justifying
  % Put here the content of the 3rd slide
\end{frame}

%%-----------------------------------------------------------------------------
\begin{frame}[t]\justifying
  % Put here the content of the 4th slide

\end{frame}

%%-----------------------------------------------------------------------------
\begin{frame}[t]{Example: an algorithm with caption}
%% This is needed if you want to add comments in
%% your algorithm with \Comment
\DontPrintSemicolon

\SetKwComment{Comment}{\# }{}
\LinesNumbered
\centering
\scalebox{.8}{%
\begin{algorithm}[H]
  \captionof{algocf}{An algorithm with caption}\label{alg:two}
\KwData{$n \geq 0$}
\KwResult{$y = x^n$}
$y \gets 1$\;
$X \gets x$\;
$N \gets n$\;
\While{$N \neq 0$}{
  \eIf{$N$ is even}{
    $X \gets X \times X$\;
    $N \gets \frac{N}{2} $ \Comment*[r]{This is a comment}
  }{\If{$N$ is odd}{
      $y \gets y \times X$\;
      $N \gets N - 1$\;
    }
  }
}
\end{algorithm}
}
\end{frame}

%%----------------------------------------------------------------------------
\begin{frame}[allowframebreaks,t]%\justifying

  \frametitle{References}
%\bibliographystyle{alpha}

\bibliographystyle{plainnat}

\bibliography{biblio.bib}

%\nocite{*}

\end{frame}

\end{document}

%%% Local Variables:
%%% mode: latex
%%% TeX-master: t
%%% End:
